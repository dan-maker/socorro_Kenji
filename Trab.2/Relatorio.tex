\documentclass{article}
\usepackage[utf8]{inputenc}
\usepackage{amsmath}
\usepackage{verbatim}
\usepackage{amsfonts}
\usepackage{graphicx}
\usepackage{listings}
%
\title{%
    2º Projeto da disciplina Estruturas de Dados II \\
     \large Análise assintótica de algoritmos de ordenação}
\author{Danillo Mendes Santiago 10414592\\Gabriel Passarelli 11218480\\ Marcelo Kenji Noda 11275359}
%
\begin{document}
%
\maketitle
%
\section{Introdução}
O propósito deste texto é analisar a complexidade de tempo de algoritmos de busca sequancial e de busca por espalhamento. Eles foram implementados na parte prática do projeto, em linguagem C de programação, e seguindo os templates fornecidos na proposta do trabalho.\par
%
O texto está dividido em seções, e há uma seção para cada algoritmo implementado. Em cada um delas, apresentamos os resultados das medições do tempo de execução de cada algoritmo, e comentamos detalher pertinentes de nossa implementação. Nessa parte, incluímos tabelas para facilitar a visualização dos dados de tempo. Como cada medição foi realizada três vezes, as tabelas contêm também medidas de disperção dos dados (o desvio-padrão). Por fim, há uma seção apresentando as conclusões desenhadas pelo grupo.
%
\section{Análise dos algoritmos de busca sequancial}
\subsection{Busca sequencial simples}
%
%
%
\subsection{Busca sequencial com método \textit{mover-para-frente}}
%
%
%
\subsection{Busca sequencial com método de \textit{transposição}}
%
%
%
\subsection{Busca sequencial com índice primário}
%
%
%
\section{Análise dos algoritmos de busca por espalhamento}
\subsection{Hash com \textit{overflow progressivo}}
%
%
%
\subsection{Hash com \textit{hash duplo}}
%
%
%
\subsection{Hash aberto com lista encadeada não ordenada}
%
%
%
%%%%%%%%%%%%%%%%%%%%%%%%%%%%%%%%%%%%%%%%%%%%%%%%%%%%%%%%%%%%%%%%
%% MODELO PARA A TABELA %%
\begin{comment}
\begin{table}
    \begin{tabular}{c|c|c|c|c|c}
        n = & $10^{1}$ & $10^{2}$ & $10^{3}$ & $10^{4}$ & $10^{5}$ \\ 
        \hline
        Vetor aleatório & $5\cdot 10^{-6}$ & $4.9\cdot 10^{-5}$ & $3.3\cdot 10^{-4}$ & $3.81\cdot 10^{-3}$ & $4.49\cdot 10^{-2}$ \\
        \hline
        Vetor crescente & $2.0\cdot10^{-6}$ & $1.5\cdot 10^{-5}$ & $2.56\cdot 10^{-4}$ & $3.39\cdot 10^{-3}$ & $3.93\cdot 10^{-2}$\\
        \hline
        Vetor decrescente & $2\cdot10^{-6}$ & $1.6\cdot 10^{-5}$ & $2.52\cdot 10^{-4}$ & $3.25\cdot 10^{-3}$ & $3.77\cdot 10^{-2}$\\
        \hline
        Média & $3\cdot 10^{-6}$ & $2.67\cdot 10^{-5}$ & $2.79\cdot10^{-4}$ & $3.48\cdot 10^{-3}$ & $4.19\cdot 10^{-2}$ \\
        \hline
        Desvio Padrão & $1.41\cdot 10^{-6}$ & $1.58\cdot 10^{-5}$ & $3.59\cdot 10^{-5}$ & $2.42\cdot 10^{-4}$ & $4.88\cdot 10^{-3}$ \\
    \end{tabular}
    \caption{Medidas de tempo para o RadixSort em segundos}
\end{table}\par
\end{comment}
%%%%%%%%%%%%%%%%%%%%%%%%%%%%%%%%%%%%%%%%%%%%%%%%%%%%%%%%%%%%%%%%%%%%%%%%
\section{Conclusão}
%
%
%
\end{document}
